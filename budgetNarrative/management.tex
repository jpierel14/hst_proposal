\forceindent The first stage of this project is to produce an open-source software
package called Supernova Time Delays ({\tt SNTD}). The software is being
developed as an open-source project, and is already publicly
accessible on github (\url{https://github.com/jpierel14/sntd}).  {\tt SNTD}
relies on two other public software packages: Python Curve Shifting
({\tt PyCS}; \citealt{Tewes:2013a}) a tool developed for measurement of
quasar time delays, and the SN light curve analysis package {\tt sncosmo}
(\citep{Barbary:2014}). Our further development of {\tt SNTD} will proceed in three steps: 1)
Integrate SNCosmo and PyCS, producing a tool that can model
SN light curve data with the abilities present in either software
package; 2) Extend and optimize the lensing and microlensing algorithm
present in PyCS for SNe; 3) Simulate a large number of multiply-imaged
SN light curves using SNCosmo and test the ability of SNTD to
simultaneously determine SN and lensing parameters. This work will be
done primarily by graduate student PI Roberts-Pierel, under the
guidance of Co-I Rodney at USC.  We anticipate this will require 4-6
months of full time effort from the PI.

In parallel with the software development, we will improve the
photometry of the two multiply-imaged SNe (Refsdal and iPTF16geu) by
1) reprocessing the HST images using the most up-to-date AstroDrizzle
software; 2) developing new time-variable PSF models, based on stellar
sources within each image; and 3) deriving new photometric time
series, using multiple photometry packages to extract photometry from
both the template-subtracted difference images and directly from the
FLT files. Although SN Refsdal appeared in the Hubble Frontier Fields,
for this work it is necessary to reprocess the data separately from
the official HFF products, because we need to break it up into
separate epochs.  This component will require 4-6 months of part-time
work from the two Co-I's, supplemented by additional effort from the
grad student PI.

For the conclusion of this project, we will use the new SNTD package
to measure improved gravitational lensing parameters for SN Refsdal
and iPTF16geu.   We anticipate that the
SNTD package and the results of the new lensing analyses will be
separately published within 12 months of the start of this project.
